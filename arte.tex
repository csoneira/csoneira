\documentclass[a4paper,12pt]{article}
\usepackage[spanish]{babel}
\usepackage[utf8]{inputenc}

\usepackage[T1]{fontenc}
\usepackage{graphicx}
\usepackage{color}
\usepackage{anysize}
\usepackage{multicol}
\usepackage{bm}
\usepackage{textcomp}
\usepackage{eurosym}
\usepackage{amsthm}
\usepackage{amsmath,amsfonts}
\usepackage{lineno}
\usepackage{amssymb}
\usepackage{xcolor}
\usepackage{booktabs}
\usepackage{graphicx}
\usepackage{float}

\marginsize{3cm}{3cm}{3cm}{3cm}
\parindent=3mm
\parskip=3mm
\renewcommand{\baselinestretch}{1.5}

\title{Reflexiones sobre el arte}
\author{C. Soneira}
\date{2018}

\begin{document}
\maketitle
Hace un par de días mi amigo Iago dio una pequeña charla acerca del concepto y la naturaleza del arte. Este hecho sumado a mi reciente suscripción a un canal de YouTube de un doctor en Bellas Artes llamado Antonio García Villarán ha suscitado en mí ciertas inquietudes relacionadas con estos temas humano-filosóficos que se suman al eterno debate que también me recorre acerca de la naturaleza misma de la Física, su relación con las Matemáticas respecto a una posible axiomatización y sus límites en la descripción del Universo.

Respecto al arte, o Arte, surgen diversas preguntas: ¿cuál es su definición?, ¿es posible medir de una forma objetiva la calidad de una obra artística y, a su vez y como consecuencia, seprar el verdadero arte de esas obras realizadas sin talento o simplemente oportunistas y demagógicas (lo que Villarán denomina \textit{hamparte})? ¿Si todo vale, cómo distinguir lo bueno de lo malo? O una pregunta aún más importante: ¿es realmente necesario crear esta distinción? ¿Hace falta separar el arte según su calidad? Trataré de reflexionar sobre estas ideas en las siguientes líneas.

En primer lugar quiero aclarar el concepto, intentar buscar una idea lo más rica y amplia posible para trabajar con ella más adelante. Para Iago, el arte es todo aquello que transmita un concepto o genere una emoción. Esta definición no entra en la intención de la obra, en su valor económico o ni siquiera en la importancia del creador: la Naturaleza es arte, Monet es arte, unos papeles colocalos sin quererlo sobre una mesa son arte; el ser humano es un ser sensible que se conmueve de forma natural al encontrarse con lo que le rodea, por esta definición, el mundo en sí es arte (no tiene que ser un sentimiento compartido por la sociedad, sino la unión de cada experiencia individual). ¿Cuál es el problema entonces? Me da la impresión de que el estudio deja de importar: conocer una obra artística \textit{tradicional} consiste en estudiar su contexto, su técnica, sus temas y cómo los transmite. Si todo es arte, ¿para qué sirve este estudio concreto? En mi opinión, tiene que ser para entenderse más, para disfrutarla más y, por tanto, para llevarte a una experiencia sensible más elevada. Obras de arte realizadas sin talento son incapaces de crear este efecto: mi interpretación (al piano) de la Gymnopèdie 1 de Erik Satie es arte según la definición de Iago, pero es cutre y tiene muchos errores, por lo que no produce ese sentimiento de belleza al que puede llegar ese tema tocado correctamente.

Por ello, propongo a continuación dos ideas que podrían, no caracterizar ni delimitar, pero sí clasificar el arte en la definición de Iago: una obra artística (definida como un cuerpo físico o idea delimitado del resto de la realidad en cuyos límites adquiere un máximo significado) puede decirse \textit{mejor} que otra si, a través de su estudio o de su simple naturaleza, aspira a un nivel experiencial e intelectual más elevado que otra. Otra posible clasificación, que se podría incluso mezclar con la otra, sería decir que una obra artística es, denuevo, \textit{mejor} que otra si, entendiendo a qué aspira, desarrolla mediante buen hacer técnico o formal o incluso en su mero acercamiento a la belleza (lo cual extiende la definición a obras sin artista creador, como las que se encuentran en la Naturaleza) una obra artística que explota al máximo los recursos de los que dispone para realizarse. Picasso desarrolló muchos movimientos artísticos con más éxito (\textit{mejor}) que sus propios creadores: utilizando su talento lleva la obra a su máximo nivel expresivo, por ello podría considerarse mejor. Defino talento como [...]

\subsubsection*{Apéndice}
En su momento, tal y como se acaba de comprobar, dejé este texto en el tintero. Lo he mantenido intacto por ser el primero que hice y, aunque la postura que ahí defiendo sería fácilmente refutable por mi testaruda persona actual, me hace ilusión casi arqueológica guardarlo. Solo quiero añadir en este apéndice lo que yo creo que entendía por \textit{talento} cuando lo mencionaba en este texto pues, aunque no se trata de un concepto central, sí es algo bastante recurrente en la discusión que articulo. Así pues, \textit{talento} no es otra cosa que capacidad, o poder, para llevar a cabo la tarea artística. Tal y como me refiero a dicho concepto durante el texto me parece claro que no lo refiero como una capacidad innata o incluso de origen divino, sino simplemente como el conjunto de destrezas en la técnica artística que permite ejecutar la creación. Quede dicho, pues.

\end{document}
